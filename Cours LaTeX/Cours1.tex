\documentclass[francais, 10pt]{article}

%%% PACKAGE %%%
\usepackage[francais]{babel}
\usepackage[utf8]{inputenc}
\usepackage{lmodern}
\usepackage[T1]{fontenc}
%%%%%%%%%%%%

%%% PREAMBULE %%%
%Ceci est le pr\'eambule du document.


\title{Cours LaTeX sur les articles}
\author{Un auteur}
\date{18 novembre 2016}

\begin{document}

\maketitle

\tableofcontents

\abstract{Dans ce cours, le lecteur apprendra facilement à se familiariser avec cet outil merveilleux qu'est \LaTeX}

\part{Titre de la partie 1}
	\section{Une section}
		La section
		\subsection{Une sous-section}
			La sous-section
				\paragraph{}
					Ici mon paragraphe
					\subparagraph{}
						Là mon sous paragraphe
				\subsubsection{Une sous-sous-section}
				La sous-sous-section

\part{Titre de la partie 2}
	\section{Une section}
		La section
		\subsection{Une sous-section}
			La sous-section
			\subsubsection{Une sous-sous-section}
				La sous-sous-section
	\section*{Section non numérotée}

\part{Les environnements}
	\section{Les listes}
		\subsection{Les listes non triées}
			\begin{itemize}
				\item \LaTeX permet de faire des listes
					\item Les listes sont des suites de points
				\item la macro \verb|\item| se place en début de chacun des points
			\end{itemize}
 		\subsection{Les listes triées}
			\begin{enumerate}
				\item \LaTeX permet de faire des listes
					\item Les listes sont des suites de points
				\item la macro \verb|\item| se place en début de chacun des points
			\end{enumerate}
		\subsection{Listes récursives}
			\begin{enumerate}
				\item \LaTeX permet de faire des listes
				\begin{itemize}
					\item \LaTeX permet de faire des listes
					\item Les listes sont des suites de points
					\item la macro \verb|\item| se place en début de chacun des points
				\end{itemize}
				\item Les listes sont des suites de points
				\begin{enumerate}
					\item \LaTeX permet de faire des listes
					\item Les listes sont des suites de points
					\begin{enumerate}
						\item \LaTeX permet de faire des listes
						\item Les listes sont des suites de points
						\item la macro \verb|\item| se place en début de chacun des points
					\end{enumerate}
					\item la macro \verb|\item| se place en début de chacun des points
				\end{enumerate}
				\item la macro \verb|\item| se place en début de chacun des points
			\end{enumerate}
			
			\begin{itemize}
				\item \LaTeX permet de faire des listes
				\item Les listes sont des suites de points
				\begin{itemize}
					\item \LaTeX permet de faire des listes
					\item Les listes sont des suites de points
					\begin{itemize}
						\item \LaTeX permet de faire des listes
						\item Les listes sont des suites de points
						\item la macro \verb|\item| se place en début de chacun des points
					\end{itemize}
					\item la macro \verb|\item| se place en début de chacun des points
				\end{itemize}
				\item la macro \verb|\item| se place en début de chacun des points
			\end{itemize}

	\section{Les équations}
		\subsection{Equation rapide}
			\begin{math} \alpha = \beta^2 + \gamma_1 + \delta_{1 + 2 + 3 + 4 + 5} \end{math}
			\\
			\verb|\begin{math}| et \verb|\end{math}| peuvent être remplacés par \$ ... \$
		\subsection{Equation détachée}
			\begin{displaymath} \sum_{ \alpha = \beta^2 + \gamma_1 }^{  \alpha = \infty } {\alpha^\beta + 1}\end{displaymath}
			\verb|\begin{displaymath}| et \verb|\end{displaymath}| peuvent être remplacés par \$\$ ... \$\$
	
	\section{What You See Is What You Get}
		\subsection{Environnement}
		\begin{verbatim}
			int carre(int a){
			   return (a * a);
			}
		\end{verbatim}
	
		\subsection{En ligne}
			\verb+int carre+
	
		\subsection{mode *}
		\begin{verbatim*}
			enfin !!!        du bon vieux what you see is what you get !!!
		\end{verbatim*}
		
\part{Emphase}

	\section{Emphase de Style}
		On peut mettre en valeur un \emph{mot}
		\\
		Le texte peut être mis $\textbf{en gras}$ , $\textit{en italique}$, \dots
		\\
		Les tailles de texte vont de {\Huge Huge} à {\tiny tiny}
	\section{Notes de bas de page}
		Et même des notes de bas de page \footnote{Illustration}

\part{Macro}
	\newcommand{\Cact}{
		\mathcal{C}_{act}
	}
	
	$\Cact$
	


\end{document}