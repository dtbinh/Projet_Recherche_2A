\documentclass[francais, 10pt]{article}

%%% PACKAGE %%%
\usepackage[francais]{babel}
\usepackage[utf8]{inputenc}
\usepackage{lmodern}
\usepackage[T1]{fontenc}
%%%%%%%%%%%%

\begin{document}
\section{Objectif de la recherche}
Il faut faire en sorte de rendre le travail reproductible. Pour cela, on doit raconter explicitement le problème et poser un modèle (qui doit se baser sur les travaux d'autres chercheurs et être justifié) par-dessus pour le résoudre. Ce modèle doit être clairement établi et amener ensuite à la simulation qui va permettre d'obtenir des résultats sur le modèle. Le modèle permet de juger si la simulation lui répond correctement.

Quel est le problème qu'on veut faore ? Qui l'a déjà traité ? Cela a-t-il un intérêt ?

\section{Cas étudié}
On cherche à attaquer un parc de machine avec des virus (Quoi, pourquoi et intérêt, comment ?).  Pour définir le modèle associé, il faut expliquer quel est l'environnement, et les agents qui interviennent. Ensuite, on doit être capable de donner la procédure d'attaque des virus, et l'enchaînement des phases qui constituent leur cycle de vie. Le problème va nous donner éventuellement les contraintes que l'on souhaite imposer à notre modèle ainsi que les ressources disponibles. Les résultats obtenus par la simulation doivent être fiables pour que les autres puissent reproduire l'expérience et obtenir les mêmes résultats. 

Quel est l'environnement ? Quels sont les agents ? Quels sont les différentes étapes du problèmes ? Quels sont les fonctions utilisés par les agents ? 

\end{document}