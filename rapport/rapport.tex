% !TEX TS-program = pdflatex
% !TEX encoding = UTF-8 Unicode

% This is a simple template for a LaTeX document using the "article" class.
% See "book", "report", "letter" for other types of document.

\documentclass[11pt]{article} % use larger type; default would be 10pt

\usepackage[utf8]{inputenc} % set input encoding (not needed with XeLaTeX)

%%% Examples of Article customizations
% These packages are optional, depending whether you want the features they provide.
% See the LaTeX Companion or other references for full information.

%%% PAGE DIMENSIONS
\usepackage{geometry} % to change the page dimensions
\geometry{a4paper} % or letterpaper (US) or a5paper or....
% \geometry{margin=2in} % for example, change the margins to 2 inches all round
% \geometry{landscape} % set up the page for landscape
%   read geometry.pdf for detailed page layout information

\usepackage{graphicx} % support the \includegraphics command and options

% \usepackage[parfill]{parskip} % Activate to begin paragraphs with an empty line rather than an indent

%%% PACKAGES
\usepackage{booktabs} % for much better looking tables
\usepackage{array} % for better arrays (eg matrices) in maths
\usepackage{paralist} % very flexible & customisable lists (eg. enumerate/itemize, etc.)
\usepackage{verbatim} % adds environment for commenting out blocks of text & for better verbatim
\usepackage{subfig} % make it possible to include more than one captioned figure/table in a single float
% These packages are all incorporated in the memoir class to one degree or another...

%%% HEADERS & FOOTERS
\usepackage{fancyhdr} % This should be set AFTER setting up the page geometry
\pagestyle{fancy} % options: empty , plain , fancy
\renewcommand{\headrulewidth}{0pt} % customise the layout...
\lhead{}\chead{}\rhead{}
\lfoot{}\cfoot{\thepage}\rfoot{}

%%% SECTION TITLE APPEARANCE
\usepackage{sectsty}
\allsectionsfont{\sffamily\mdseries\upshape} % (See the fntguide.pdf for font help)
% (This matches ConTeXt defaults)

%%% ToC (table of contents) APPEARANCE
\usepackage[nottoc,notlof,notlot]{tocbibind} % Put the bibliography in the ToC
\usepackage[titles,subfigure]{tocloft} % Alter the style of the Table of Contents
\renewcommand{\cftsecfont}{\rmfamily\mdseries\upshape}
\renewcommand{\cftsecpagefont}{\rmfamily\mdseries\upshape} % No bold!

%%% END Article customizations

%%% The "real" document content comes below...

\title{Rapport Projet Recherche}
\author{François Torregrossa}
%\date{} % Activate to display a given date or no date (if empty),
         % otherwise the current date is printed 

\begin{document}
\maketitle

\section{Objectifs du projet}

Nous voulons simuler l'attaque de virus sur un individu protégé par des anticorps. 

\subsection{Environnement}

L'environnement est composé de cellules pouvant être détruites par les virus. Ces cellules se régénèrent après un certains temps et redonnent de l'energie aux virus. Les agents se déplacent dans cet environnement.

\subsection{Agents}

On distingue trois types d'agents :
\begin{itemize}
	\item Les cellules qui se régénèrent après un certain temps, qui peuvent être détruites.
	\item Les anticorps qui se déplacent aléatoirement dans l'environnement et détruisent les virus.
	\item Les virus qui tuent les cellules et cherchent à éviter les anticorps. Leur but est de tuer leur hôte.
\end{itemize}

\subsection{Conditions de victoire}

On considère que les virus gagnent la simulation s'ils survivent assez longtemps dans un nombre suffisant, c'est-à-dire s'ils arrivent à faire descendre le nombre de cellules en-dessous d'un seuil pendant assez longtemps.

\section{Outils utilisés}

\subsection{Programmation}

On programme avec NetLogo, qui fournit une interface graphique simple d'utilisation et est très bien documenté.

\subsection{Algorithme}

On va utiliser un algorithme qui se base sur la représentation du système immunitaire de Jerne. Cet algorithme doit servir aux virus à lutter intelligemment contre les attaques des anticorps.
\\
En détail l'algorithme sert à sélectionner des fonctions prédéfinies, qui répondent à des besoins spécifiques. Ces besoins représentent les antigènes du système immunitaire algorithmique, les fonctions quant à elles, sont associés aux anticorps. Comme dans un système immunitaire de Jerne, les fonctions ou anticorps se stimulent entre eux. \`A cela s'ajoute la stimulation des anticorps. En faisant la synthèse des stimulations, on est capables de calculer la concentration des anticorps (ou fonctions) dans ce système, et ainsi, en choisissant la fonction de concentration la plus élevée, sélectionner la réponse adaptée au besoin initial.
\\
Le calcul des concentrations se fait au fur et à mesure de l'observation du comportement des fonctions. Alors, on détermine si une fonction est positive (stimulation au prochain choix) ou négative (inhibition au prochain choix) . L'inhibition aura pour effet de stimuler l'appel aux autres fonctions, quant à  la stimulation, elle entraînera une auto-stimulation.
\\
J'ai pu distinguer deux cas d'utilisation du système immunitaire de Jerne.
\begin{itemize}
	\item Premièrement, soit les fonctions ont des réponses facilement caractérisables. C'est-à-dire qu'on peut facilement juger si la réponse qu'elle a donnée après avoir été sélectionné pour un besoin, a été correcte ou non. Dans ce cas les besoins spécifiques sont clairement définie, et on peut relier facilement chaque besoin à une fonction et favoriser la stimulation de cette fonction.
	\item L'autre cas sera celui des fonctions difficilement caractérisables. Soit parce que la qualité des réponses des fonctions est difficilement estimable, soit parce qu'on n'en a aucune idée (ex.: seuil à partir duquel on veut changer de comportement). Dans ce cas, la meilleure chose qu'on puisse faire est de mettre un seul antigene, et regarder si à l'utilisation d'une fonction cet antigène disparaît ou s'intensifie.
\end{itemize}

\subsection{\'Equation}
\begin{equation}
	\frac{dA_i(t)}{dt} = \left(\alpha\cdot\frac{1}{N}\cdot\sum_{j=1}^{N}m_{ij}a_j(t)-\alpha \cdot \frac{1}{M}\cdot\sum_{k=1}^{M}m_{ik}a_k(t)+\beta m_i-k_i\right)\cdot a_i(t)
	\label{equation21}
\end{equation}
où : 
\begin{itemize}
	\item $a_i$ représente la concentration de la fonction $i$, les deux sommes de gauche représente respectivement la stimulation par les autres fonctions et l'inhibition
	\item $m_{ij}$ représente la stimulation de la fonction j par la fonction i. C'est ce terme précisément qui est calculé grace aux matrices précédentes.
	\item $m_i$ représente la stimulation directe par un virus (i.e. ce terme vaut 1 lorsque le système immunitaire appelle la fonction correspondante et 0 sinon)
	\item $k_i$ simule la mort naturelle des anticorps (peut être interprété comme un pourcentage de cellules qui meurent entre deux phases)
	\item $M$ est le nombre de fonction inhibant $i$ et $N$ le nombre de fonction stimulant $i$. Dans notre cas $M=N$, mais il est possible d'imaginer certains cas où les fonctions ne sont pas forcément toutes inhibitrices les unes des autres.
	\item $A_id$ est la concentration non normalisée  $a_i$, il faut alors le passer dans la fonction sigmoid : $f(x) = \frac{1}{1 + exp(0.5 - x)}$
\end{itemize}

Pour calculer $m_{ij}$ en présence du virus $k$ on applique la formule suivante \cite{ref}:
\begin{equation}
	m_{ij} = \frac {penalty\_matrix[i][k] + reward\_matrix[j][k]}{specific\_matrix[i][k] + specific\_matrix[j][k]}
\end{equation}
Plus clairement :
\begin{itemize}
	\item Au numérateur on somme la quantité de fois que la fonction $i$ n'a pas été efficace avec le virus k avec la quantité de fois que la fonction $j$ a été efficace avec le virus $k$.
	\item Au dénominateur, il y a le nombre de fois que chaque fonction a été appelée en présence du virus $k$.
\end{itemize}


\section{Implémentation}

On organise le système de décision des virus.

\subsection{Premier niveau}
Ce niveau est chargé de la décision du comportement global des virus. Il doit être capable de décider si les virus doivent : se multiplier, s'étendre, ou se réduire.

\subsection{Second niveau}
Ce niveau est individuel à chaque virus. Selon le mode choisit au dessus, les virus vont recevoir plusieurs fonctions.
\begin{itemize}
	\item Multiplication (Critère nombre de virus):
		\begin{itemize}
			\item Se diviser (division cellulaire)
			\item Se cacher (se cache des anticorps)
		\end{itemize}
	\item Expansion (Critère nombre de cellules vivantes):
	 	\begin{itemize}
	 		\item Attaquer (procure de l'energie au virus et leur permet de vivre plus longtemps)
	 		\item Bouger en tentant d'éviter (eviter les anticorps)
	 	\end{itemize}
	\item Réduction (Critère nombre d'anticorps) :
		\begin{itemize}
			\item S'immobiliser
			\item S'entretuer
			\item Mourir de faim 
		\end{itemize}
\end{itemize}


\nocite{ref2}
\bibliographystyle{plain}
\bibliography{biblio}


\end{document}
